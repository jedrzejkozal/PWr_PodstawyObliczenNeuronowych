\documentclass{article}

\usepackage{polski}
\usepackage[utf8]{inputenc}
\usepackage{graphicx}
\usepackage{xcolor}
\usepackage{float}
\usepackage{caption}
\usepackage{array}
\usepackage{pbox}

\newcommand\tab[1][1cm]{\hspace*{#1}}

\title{Dokumentacja projektowa}
\date{2018-03-18}
\author{Jędrzej Kozal}

\begin{document}

\begin{titlepage}
	\centering
	\includegraphics[width=0.25\textwidth]{logo_pol_wroclaw.png}\par\vspace{1cm}
	{\scshape\LARGE Politechnika Wrocławska \par}
	\vspace{1cm}
	{\scshape\Large Podstawy Obliczeń Neuronowych\par}
	\vspace{1.5cm}
	{\huge\bfseries Sieci radialne w predykcji procesów medycznych - choroba Parkinsona \par}
	\vspace{2cm}
	{\Large\itshape Filip Guzy\par}
	{\Large\itshape Jędrzej Kozal\par}

	\vfill
	prowadzący\par
	Dr inż.~Edward \textsc{Puchała}

	\vfill

% Bottom of the page
	{\large \today\par}
\end{titlepage}

\tableofcontents
\newpage


\section{Wstęp}

\subsection{Cel projektu}

Celem projektu było zbadanie skuteczności radialnych sieci neuronowych w predykcji przebiegu choroby Parkinsona.

\subsection{Wykorzystane narzędzia}

Do realizacji projektu wykorzystano środowisko Matlab, w tym Deep Learning Toolbox zawierający funkcje niezbędne do tworzenia radialnych sieci neuronowych oraz dane ze zbioru uczącego UCI Oxford Parkinson's Disease Telemonitoring Dataset, udostępnionego przez autorów do celów naukowo-badawczych \cite{zbioruczacy}. 

\subsection{Aspekt medyczny}

Parkinson jest zwyrodnieniową chorobą ośrodkowego układu nerwowego. Występuje ona najczęściej u osób starszych, głównie mężczyzn i rozwija się przez kilkanaście lat. Jej objawami są między innymi: osłabienie, zmęczenie, spowolnienie ruchowe, drżenie mięśni czy też kłopoty z wykonywaniem codziennych czynności, takich jak wstawanie, mycie, jedzenie, ubieranie się.  Objawy te oraz sposób rozwoju choroby pozwalają na stworzenie systemów predykcyjnych umożliwiających ciągłe monitorowanie stanu zdrowia pacjentów. Zbiór uczący użyty w projekcie wykorzystuje pomiary głosu pacjentów wykonane przez urządzenie telemedyczne służące do śledzenia przebiegu choroby w sześciomiesięcznym przedziale czasu. Dla pojedynczego pacjenta zawiera on parametry takie jak: identyfikator, wiek oraz płeć, interwały czasowe wykonanych pomiarów, szesnaście parametrów opisujących nagrany dźwięk, a także parametry wynikowe UPDRS (Unified Parkinson's Disease Rating Scale) opisujące postęp choroby. W zbiorze znajdują się dane zawierające informację o 42 różnych pacjentach, po około 140 pomiarów dla pojedynczej osoby, co łącznie daje około 5000 rekordów.

\section{Zarys podstaw teoretycznych}



\section{Osiągnięte rezultaty}

1) 13.10.2018 \\


number\_of\_patients = 42\\
traning\_set\_size = 30\\
test\_set\_size = 12;\\

eg = 0.1; % sum-squared error goal\\
sc = 1;    % spread constant

Zmniejszenie liczby parametrów powoduje skrócenie czasu uczenia sieci, ale dalej pozostaje overfiting. Dodanie wieku i płci do parametrów na wejściu sieci spowodowało znaczne przyśpieszenie uczenia sieci. Dla parametrów: test\_time, Jitter(%), Shimmer(dB) sieć nie osiągała dostatecznie niskich wartości funkcji celu mimo dużej ilości epok. Po dodaniu age i sex sieć do parametrów wejściowych uzyskano znacząco przyśpieszono proces uczenia sieci. Zmniejszenie progu kwadratowej funkcji celu z 0.1 do 0.05 nie dało znaczących rezultatów.

Wyłączenie czasu z wektora wejściowego i zmiana wybierania zbioru testowego na 24 pierwsze próbki z nieposorotwanych danych z bazy UCI spowodowało podwyższenie progu wartości średnio kwadratowego błędu do 8, co znacząco pogorszyło dopasowanie nawet dla zbioru uczącego.





\begin{figure}
\centering
	\includegraphics[width=0.90\textwidth]{fig1.jpg}\par\vspace{1cm}
\caption{Rezultaty dla zbioru uczącego}
	\label{fig:features}
\end{figure}

\begin{figure}
\centering
	\includegraphics[width=0.90\textwidth]{fig2.jpg}\par\vspace{1cm}
\caption{Rezultaty dla zbioru testowego}
	\label{fig:features}
\end{figure}

\section{Wnioski}

\newpage
\begin{thebibliography}{9}

\bibitem{zbioruczacy}
A Tsanas, MA Little, PE McSharry, LO Ramig (2009) 
'Accurate telemonitoring of Parkinson’s disease progression by non-invasive speech tests', 
IEEE Transactions on Biomedical Engineering (to appear). 
\\\texttt{https://archive.ics.uci.edu/ml/datasets/Parkinsons+\\Telemonitoring}

\bibitem{zbior-uczacy1}
Little MA, McSharry PE, Hunter EJ, Ramig LO (2009), 
'Suitability of dysphonia measurements for telemonitoring of Parkinson's disease', 
IEEE Transactions on Biomedical Engineering, 56(4):1015-1022 

\bibitem{zbior-uczacy2}
Little MA, McSharry PE, Roberts SJ, Costello DAE, Moroz IM. 
'Exploiting Nonlinear Recurrence and Fractal Scaling Properties for Voice Disorder Detection', 
BioMedical Engineering OnLine, 26 June 2007

\bibitem{Osowski ujecie algorytmiczne}
Stanisław Osowski.
\textit{Sieci Neuronowe w ujęciu algorytmicznym}
Wydawnictwo Naukowo-Techniczne, Warszawa 1996

\bibitem{Osowski przetwarzanie informacji}
Stanisław Osowski.
\textit{Sieci Neuronowe do przetwarzania informacji}
Oficyna Wydawnicza Politechniki Warszawskiej, Warszawa 2006

\bibitem{Spread Constant}
Hongfa Wang, Xinai Xu, Zhejiang
Water Conservancy and Hydropower College
Determination of Spread Constant in RBF Neural Network by Genetic
Algorithm 
\textit{http://www.globalcis.org/ijact/ppl/IJACT3017PPL.pdf}

\bibitem{matlab}
Oficjalna dokumentacja Mathworks Deep Learning Toolbox. 
\\\texttt{https://www.mathworks.com/help/deeplearning/examples/radial-\\basis-approximation.html}

\bibitem{owards Data Sience}
Towards Data Science blog
\\\texttt{https://towardsdatascience.com/radial-basis-functions-neural-\\networks-all-we-need-to-know-9a88cc053448}


\end{thebibliography}
\newpage

\listoffigures
\newpage

\end{document}