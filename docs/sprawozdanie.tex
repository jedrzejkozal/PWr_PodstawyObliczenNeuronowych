\documentclass{article}

\usepackage{polski}
\usepackage[utf8]{inputenc}
\usepackage{graphicx}
\usepackage{xcolor}
\usepackage{float}
\usepackage{caption}
\usepackage{array}
\usepackage{pbox}

\newcommand\tab[1][1cm]{\hspace*{#1}}

\title{Dokumentacja projektowa}
\date{2018-03-18}
\author{Jędrzej Kozal}

\begin{document}

\begin{titlepage}
	\centering
	\includegraphics[width=0.25\textwidth]{logo_pol_wroclaw.png}\par\vspace{1cm}
	{\scshape\LARGE Politechnika Wrocławska \par}
	\vspace{1cm}
	{\scshape\Large Podstawy Obliczeń Neuronowych\par}
	\vspace{1.5cm}
	{\huge\bfseries Sieci radialne w predykcji procesów medycznych - choroba Parkinsona \par}
	\vspace{2cm}
	{\Large\itshape Filip Guzy\par}
	{\Large\itshape Jędrzej Kozal\par}

	\vfill
	prowadzący\par
	Dr inż.~Edward \textsc{Puchała}

	\vfill

% Bottom of the page
	{\large \today\par}
\end{titlepage}

\tableofcontents
\newpage


\section{Wstęp}


\newpage
\begin{thebibliography}{9}

\bibitem{matlab}
Oficjalna dokumentacja Mathworks Deep Learning Toolbox. 
\\\texttt{https://www.mathworks.com/help/deeplearning/examples/radial-basis-approximation.html}

\bibitem{owards Data Sience}
Towards Data Science blog
\\\texttt{https://towardsdatascience.com/radial-basis-functions-neural-networks-all-we-need-to-know-9a88cc053448}


\end{thebibliography}
\newpage

\listoffigures
\newpage

\end{document}